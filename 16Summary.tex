\documentclass[12pt]{article}
\usepackage{multicol,graphicx}
\usepackage[colorlinks,breaklinks,linkcolor=red,citecolor=blue]
{hyperref} 
\def\sectionautorefname~#1\null{\S#1\null}
\usepackage{charter,amsmath,amssymb,breakurl}
\usepackage{eulervm}
\usepackage[letterpaper,margin=.75in]{geometry}
\def\equationautorefname~#1\null{(#1)\null}
\def\itemautorefname~#1\null{(#1)\null}
\title{Chapter 16 Summary}
\author{}\date{}
\let\ln\relax\DeclareMathOperator{\ln}{\mathsf{ln}}
\let\sin\relax\DeclareMathOperator{\sin}{\mathsf{sin}}
\let\arctan\relax\DeclareMathOperator{\arctan}{\mathsf{arctan}}
\let\cos\relax\DeclareMathOperator{\cos}{\mathsf{cos}}
\let\sec\relax\DeclareMathOperator{\sec}{\mathsf{sec}}
\let\min\relax\DeclareMathOperator*{\min}{\mathsf{min}}
\let\max\relax\DeclareMathOperator*{\max}{\mathsf{max}}
\let\sup\relax\DeclareMathOperator*{\sup}{\mathsf{sup}}
\let\inf\relax\DeclareMathOperator*{\inf}{\mathsf{inf}}
\let\lim\relax\DeclareMathOperator*{\lim}{\mathsf{lim}}
\everymath{\displaystyle}
\begin{document}
\maketitle
\thispagestyle{empty}

\begin{enumerate}
\item{\bf Line integrals.}
If $\mathbold{F}$ is a vector field and
$C$ is a path, we calculate $\int_C\mathbold{F}\cdot d\mathbold{r}$ by
\[\int_a^b\mathbold{F}\left(\mathbold{r}\left(t\right)\right)
\cdot\mathbold{r}'\left(t\right)\;dt\]
where $\mathbold{r}\left(t\right)$ parameterizes $C$
as $t$ ranges from $a$ to $b$.
The value $\int_C\mathbold{F}\cdot d\mathbold{r}$ represents
the {\em work} done by $\mathbold{F}$ in moving a particle along $C$.

\item If $C$ is a loop then
$\int_C\mathbold{F}\cdot d\mathbold{r}$
is denoted by
$\oint_C\mathbold{F}\cdot d\mathbold{r}$ and called the
{\em circulation} of $\mathbold{F}$ around $C$.

\item $\mathbold{F}$ is called {\em conservative} or {\em
path independent} if for any points $P,Q$ in the domain of $\mathbold{F}$
the line integral
$\int_C\mathbold{F}\cdot d\mathbold{r}$ is the same for any
path $C$ from $P$ to $Q$.

\item{\bf Fundamental theorem of line integrals.}
If $\mathbold{F}=\nabla f$ and $C$ is any curve
connecting the points $P,Q$ then
\[\int_C\mathbold{F}\cdot d\mathbold{r}=f\left(Q\right)-f\left(P\right).\]
$f$ is called a {\em potential function}
of $\mathbold{F}$ if $\mathbold{F}=\nabla f$.

\item The following are equivalent.
\begin{enumerate}
\item $\mathbold{F}$ is conservative.
\item $\oint_C\mathbold{F}\cdot d\mathbold{r}=0$
for any loop $C$.
\item $\mathbold{F}=\nabla f$ for some function $f$.
\end{enumerate}

\item A region $R$ is called {\em path connected}
if for any two points $P,Q$ in $R$ there exists
a path from $P$ to $Q$ lying entirely in $R$.

\item A region $R$ is called {\em simply connected}
if any loop in $R$ can be contracted to a point.

\item\label{Curl}
If $\mathbold{F}=\left\langle P,Q,R\right\rangle$
is a vector field in $\mathbb{R}^3$
we define the curl and divergence of $\mathbold{F}$ by
\begin{align*}
\mathsf{curl}\left(\mathbold{F}\right)
&=\nabla\times\mathbold{F}
=\begin{vmatrix}
\mathbold{i}&\mathbold{j}&\mathbold{k}\\
\frac{\partial}{\partial x}&\frac{\partial}{\partial y}
&\frac{\partial}{\partial z}\\P&Q&R 
\end{vmatrix}
=\left\langle R_y-Q_z,P_z-R_x,Q_x-P_y\right\rangle\\
\mathsf{div}\left(\mathbold{F}\right)
&=\nabla\cdot\mathbold{F}=\left\langle P_x,Q_y,R_z\right\rangle
\end{align*}
If $\mathbold{F}=\left\langle P,Q\right\rangle$ 
is a vector field in $\mathbb{R}^2$
then we define
$\mathsf{curl}\left(\mathbold{F}\right)
=Q_x-P_y$.

\item{\bf Curl test.} If the domain of $\mathbold{F}$
is path connected and simply connected then
$\mathbold{F}$ is conservative if and only if $\mathsf{curl}
\left(\mathbold{F}\right)=\mathbold{0}$.

\item{\bf Green's theorem.}
\begin{enumerate}
\item If $\mathbold{F}=\left\langle P,Q\right\rangle$ is a vector
field in $\mathbb{R}^2$, $C$ is a loop,
and $R$ is the region inside $C$, then
\begin{equation}\label{Green1}
\oint_C Pdx+Qdy
=\int_C\mathbold{F}\cdot d\mathbold{r}
=\int_R\left(Q_x-P_y\right)\;dA
\end{equation}
\item Replacing $P,Q$ in \autoref{Green1} with $-Q,P$ gives
the equivalent form
\begin{equation}\label{Green2}
\oint_C Pdy-Qdx
=\int_R\left(P_x+Q_y\right)\;dA
\end{equation}
\item The left-hand side of \autoref{Green2} gives
the {\em outward flux} of $\mathbold{F}$ over $C$
and is denoted by $\oint_C\mathbold{F}\cdot\mathbold{n}\;ds$.
\end{enumerate}

\item Suppose $S$ is a surface parameterized by
$\mathbold{r}\left(u,v\right)
=\left\langle f\left(u,v\right),g\left(u,v\right),h\left(u,v\right)
\right\rangle$ with $a\le u\le b$ and $c\le v\le d$
\begin{enumerate}
\item The area of $S$ is given by
\begin{equation}\label{Area}
\int_c^d\int_a^b\left|\mathbold{r}_u\times\mathbold{r}_v
\right|\;dudv
\end{equation}
\item Denoting $\left|\mathbold{r}_u\times\mathbold{r}_v\right|\;dudv$
by $d\sigma$ we can write \autoref{Area} as
$\int_Sd\sigma$.
\item More generally, if $\mathbold{F}\left(x,y,z\right)$
is a vector field and $\mathbold{n}\left(u,v\right)$
is a unit normal to $S$ at $\mathbold{r}\left(u,v\right)$
then the {\em flux} of $\mathbold{F}$ through $S$
is given by
\[\int_S\mathbold{F}\cdot\mathbold{n}\;d\sigma
=\int_c^d\int_a^b\left(
\mathbold{F}\left(f\left(u,v\right),g\left(u,v\right),
h\left(u,v\right)\rule{0pt}{11pt}\right)
\cdot\mathbold{n}\left(u,v\right)\rule{0pt}{12pt}\right)
\left|\mathbold{r}_u\times\mathbold{r}_v
\right|\;dudv\]
\end{enumerate}

\item{\bf Stoke's theorem.}
If $S$ is a surface with unit normal $\mathbold{n}$
and boundary $C$ oriented according to the right-hand rule
with respect to $\mathbold{n}$, then
\begin{equation}\label{Stokes}
\oint_C\mathbold{F}\cdot d\mathbold{r}
=\int_S\mathsf{curl}\left(\mathbold{F}\right)
\cdot\mathbold{n}\;d\sigma.
\end{equation}
The right-hand side of \autoref{Stokes} is often
denoted by $\int_S\nabla\times\mathbold{F}\cdot\mathbold{n}\;
d\sigma$ using the notation from \autoref{Curl}.
\end{enumerate}
\end{document}

