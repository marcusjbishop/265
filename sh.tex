\documentclass[11pt]{article}
\usepackage{booktabs}
\usepackage[euler-digits]{eulervm}
\usepackage[colorlinks=true, linkcolor=blue, breaklinks=true]{hyperref} 
\def\sectionautorefname~#1\null{\S#1\null}
\usepackage{charter,amsmath,amssymb,breakurl}
\usepackage[letterpaper,margin=.9in]{geometry}
\title{Syllabus for Math~265, Section~1}
\author{Dr Marcus Bishop, Iowa State University}
\begin{document}
\maketitle

\section{Time, place, credits}\label{Time}
We will meet at 12:10--13:00
on Mondays, Tuesdays, Thursdays, and Fridays
in Carver~2.
%The midterm exam is at 20:15--21:45
%on Wednesday 8 October in Coover~2245.
%You should plan your schedule accordingly.
Math~265 is a four credit course.

\section{Requirements, prerequisites, materials}\label{Require}
To take this course, you should have taken Math~166
or the equivalent. You will need the twelfth edition of 
\href{http://wps.aw.com/aw_thomas_calculus_series}
{\em Thomas' Calculus, Early Trancendentals}
by Thomas, Weir, and Hass (hereafter referenced as
\href{http://wps.aw.com/aw_thomas_calculus_series}{Thomas})
bundled together with a
\href{http://iastate.mylabsplus.com}{MyLabsPlus} subscription
(see \autoref{Online}).
All other learning materials will be provided through
\href{https://bb.its.iastate.edu}{Blackboard}.
You will also want a 
\href{http://en.wikipedia.org/wiki/TI-83}{TI-83} or
\href{http://en.wikipedia.org/wiki/TI-84}{TI-84}
graphing calculator. Unfortunately, we cannot allow you to use
any device with algebraic capabilities greater than those of the
\href{http://en.wikipedia.org/wiki/TI-83}{TI-83} or
\href{http://en.wikipedia.org/wiki/TI-84}{TI-84}
on quizzes and exams. However, we encourage
you to supplement your learning with a computer algebra system
of your choice.
In class I will use \href{http://www.sagemath.org}{\sf Sage},
an open-source computer algebra system that you can install on your
computer for free if you wish.

\section{Instructor information} You may visit me during my office 
hours at 9:00--12:00 on Tuesdays and Thursdays
in Carver~418. If my office hours are inconvenient for you
please email me at 
\href{mailto:mbishop@iastate.edu}{\tt mbishop@iastate.edu} or call me at 
515-294-0027 to arrange an alternate time.
You can also visit the Math Help Room (see \autoref{MathCenter}).

\section{Course Objectives}
\subsection{Geometry in Space, Vectors}
\begin{itemize}
\setlength\itemsep{-.5em}
\item Use the parallelogram law to add geometric vectors
\item Resolve geometric vectors into components parallel to coordinate axes
\item Perform the operations of vector addition and scalar multiplication,
and interpret them geometrically
\item Use the dot product to calculate the magnitude of a vector,
the angle between vectors, 
and the projection of one vector onto another
\item Find and use direction angles and direction cosines of a vector
\item Use parametric equations for plane curves and space curves
\item Use and convert between parametric and symmetric equations for a straight line
\item Find the tangent line at a point to a parametric curve;
compute the length of a parametric curve
\item Compute the velocity, unit tangent, and acceleration vectors
along a parametric curve; 
resolve acceleration into tangential and normal components and compute curvature
\item Use and interpret geometrically the standard equation for a plane
\item Use the cross product; interpret the cross product geometrically and as area of a parallelogram; 
interpret the vector triple product as the volume of a parallelopiped
\item Recognize cylinders and quadric surfaces from their Cartesian equations
\item Use cylindrical and spherical coordinates,
and convert among these two and rectangular coordinates
\end{itemize}
\subsection{Derivatives for Functions of Two or More Variables}
\begin{itemize}
\setlength\itemsep{-.5em}
\item Represent a function of two variables as the graph of a surface;
sketch level curves
\item Calculate partial derivatives and the gradient
\item Use the gradient to find tangent planes,
directional derivatives and linear approximations
\item Interpret the gradient geometrically
\item Use the Chain Rule
\item Find and classify critical points of functions, using the second derivative test
\end{itemize}
\subsection{Multiple Integrals}
\begin{itemize}
\setlength\itemsep{-.5em}
\item State the definition of the integral of a function over a rectangle
\item Use iterated integrals to evaluate integrals over planar regions,
and to calculate volume
\item Build on elementary integration techniques to evaluate
multiple integrals efficiently
\item Set up and evaluate double integrals in polar coordinates
\item Set up and evaluate integrals to compute surface area
\item Set up and evaluate triple integrals in Cartesian coordinates
\item Use double and triple integrals to compute moments, center of mass,
and moments of inertia
\item Use cylindrical and spherical coordinates; change coordinates from 
rectangular to cylindrical or spherical or the reverse
\item Set up and evaluate triple integrals in cylindrical and spherical coordinates
\item Change the order of variables in multiple integrals
\item Carry out change of variables in multiple integrals
\end{itemize}
\subsection{Vector Calculus}
\begin{itemize}
\setlength\itemsep{-.5em}
\item Calculate the curl and divergence of a vector field
\item Set up and evaluate line integrals of scalar functions
or vector fields along curves
\item Recognize conservative vector fields, and apply the fundamental 
theorem for line integrals of conservative vector fields
\item State and apply Green's Theorem
\item Set up and evaluate surface integrals; compute surface area and 
the flux of a vector field through a surface
\item Set up and evaluate integrals over parametric surfaces
\item State and apply the Divergence Theorem
\item State and apply Stokes' Theorem
\end{itemize}

\section{Online exercises}\label{Online}
The online exercises will be delivered through the
\href{http://iastate.mylabsplus.com}{MyLabsPlus} system.
Your subscription to 
\href{http://iastate.mylabsplus.com}{MyLabsPlus}
comes with the purchase of your
\href{http://wps.aw.com/aw_thomas_calculus_series}{textbook}.
However, if you purchased the book from some other source,
then you can purchase a 
\href{http://iastate.mylabsplus.com}{MyLabsPlus}
subscription at the bookstore or directly at the website below.
To access
\href{http://iastate.mylabsplus.com}{MyLabsPlus}
sign in at
\begin{center}
\href{http://iastate.mylabsplus.com}{\tt http://iastate.mylabsplus.com}
\end{center}
with your NetID and password.
You should also run the
\href{https://www.mathxl.com/BrowserCheck/BrowserCheck.aspx?appproductid=3&courseid=2744761&handler_urn=pearson%2fmlp_mml_xl%2fslink%2fx-pearson-mlp_mml_xl&productid=ccng}{Browser Checker}
to ensure that all the exercises will appear correctly.
You can email \href{mailto://mathmlp@iastate.edu}{\tt mathmlp@iastate.edu}
in case of any problems.

In addition to providing your online assignments
\href{http://iastate.mylabsplus.com}{MyLabsPlus}
also contains a number of useful resources, not the least of which is an
electronic copy of the entire text, which could obviate
the somewhat strenuous task of transporting the book.

You will have 36~online assignment corresponding
approximately with the sections of 
\href{http://wps.aw.com/aw_thomas_calculus_series}{Thomas}
that we will cover. Each online assignment will be available while we cover the
corresponding section in class and must be completed
before 23:59 of the date shown on the assignment, which
is usually one class period after we cover the section in class.
In this way you can raise questions about
the assignment in class before completing the assignment.

\section{Written assignments}\label{Written}
There will be a written assignment due on {\em every 
Friday} that classes are held
except 16 January and 1 May
but including Fridays of weeks
in which quizzes and exams occur.
These assignments will be collected at the beginning
of class and will be graded and returned to you shortly thereafter.
These assignments together with the online exercises
(see \autoref{Online}) comprise the core of the course.
In fact, the course is 
designed in such a way that if you complete each assignment and 
correct the mistakes your make on them,
then you will be very well prepared for all the quizzes and exams.

\section{Projects}\label{Projects}
Every student will complete an project, either individually
or in small groups. The results of the projects
will either be submitted in written form or presented
in class, depending on the nature of the project.
The projects will normally be provided by the instructor
and will depend on the student's
talents and interests. Projects will generally deal with the content of the course,
but might also deal with other areas of mathematics.

\section{Calculation of grades}\label{Assessment}
Your grade will be determined by the results of 
36 online assignments (see \autoref{Online}),
four quizzes, two hour exams, one midterm exam, and one 
final exam.
In order to accommodate unforeseen events that would cause you
to pass up the opportunity to submit assignments on time,
we will exclude from your final grade
the two written assignments with the lowest scores among
all your written assignments.
Your final grade will be determined from the total
number of points you receive in the course using the schedule below,
where the final exam accounts for 200~points,
the hour and midterm exams for 150~points each,
the quizzes for 100~points,
and the online assignments for 100~points (see \autoref{Online}),
the written assignments for 100~points (see \autoref{Written}),
and the project for 50~points (see \autoref{Project}).
You will receive an A, B, C, D, F if your total lies in the range
850--1000, 750--849, 650--749, 550--649, 0--549 respectively.
All the scores mentioned above will be available to you in
\href{https://bb.its.iastate.edu}{Blackboard}. Using your scores
and the formula above, you should be able to calculate your own final grade,
and even estimate your final grade midway through the semester.
It is therefore unnecessary for you to ever ask the instructor
what your grade in the course is, which we kindly ask you to refrain from
doing.

We emphasize that the grades will be calculated in the manner described
above, not ``curved''.
However, in the unlikely event that an exam or quiz question is determined
to be unfair or overly difficult, it will be dropped from the total score of
every student.

\section{Expectations, suggestions} Naturally we expect you to attend 
class meetings, complete online assignments on 
time, prepare for quizzes and exams, and participate in classroom 
activities. We also strongly encourage you to read
\href{http://wps.aw.com/aw_thomas_calculus_series}{Thomas}
in addition to attending lectures.
In addition to being generally prudent to 
supplement your learning materials, reading the text has a number of 
advantages over attending lectures alone. Namely, the text is beautifully 
typeset and edited by professionals, very clearly written with readers 
of exactly your level in mind, and virtually free of mistakes.

\section{Course schedule}\label{Schedule} While the exact subject
matter to be covered in class shown in the following schedule is
subject to change slightly, the quizzes and exams will be conducted
on {\em exactly} the dates shown.

\begin{tabular}{c|cl|cl|cl|cl}
&\multicolumn{2}{c|}{\bf Monday}
&\multicolumn{2}{c|}{\bf Tuesday}
&\multicolumn{2}{c|}{\bf Thursday}
&\multicolumn{2}{c}{\bf Friday}\\
{\bf Week}&{\bf Date}&{\bf In class}
&{\bf Date}&{\bf In class}&{\bf Date}&{\bf In class}
&{\bf Date}&{\bf In class}\\\toprule
1&12 Jan&&13 Jan&\S12.1&15 Jan&\S12.2&16 Jan&\S12.3\\\midrule
2&19 Jan&Holiday&20 Jan&\S12.4&22 Jan&\S12.5&23 Jan&\S12.5, {\bf Quiz 1}\\\midrule
3&26 Jan&\S12.6&27 Jan&\S13.1&29 Jan&\S13.1&30 Jan&\S13.2\\\midrule
4&2 Feb&\S13.3&3 Feb&\S13.4&5 Feb&Review&6 Feb&{\bf Exam 1}\\\midrule
5&9 Feb&\S13.5&10 Feb&\S13.6&12 Feb&\S14.1&13 Feb&\S14.2\\\midrule
6&16 Feb&\S14.3&17 Feb&\S14.3&19 Feb&\S14.4&20 Feb&\S14.5, {\bf Quiz 2}\\\midrule
7&23 Feb&\S14.5&24 Feb&\S14.6&26 Feb&\S14.6&27 Feb&\S14.7\\\midrule
8&2 Mar&\S14.9&3 Mar&\S14.10&5 Mar&&6 Mar&\\\midrule
9&9 Mar&\S15.1&10 Mar&\S15.2&12 Mar&\S15.2&13 Mar&\S15.3 \\\midrule
10&23 Mar&\S15.4&24 Mar&\S15.5&26 Mar&\S15.5&27 Mar&\S15.6, {\bf Quiz 3}\\\midrule
11&30 Mar&\S15.7&31 Mar&\S15.7&2 Apr&\S16.1&3 Apr&\S16.2\\\midrule
12&6 Apr&\S16.2&7 Apr&\S16.3&9 Apr&Review&10 Apr&{\bf Exam 3}\\\midrule
13&13 Apr&\S16.3&14 Apr&\S16.4&16 Apr&\S16.4&17 Apr&\S16.5\\\midrule
14&20 Apr&\S16.5&21 Apr&\S16.6&23 Apr&\S16.7&24 Apr&\S16.7, {\bf Quiz 4}\\\midrule
15&27 Apr&\S16.8&28 Apr&\S16.8&30 Apr&Review&1 May&Review\\\midrule
\end{tabular}

\section{Math Help Room}\label{MathCenter}
We strongly encourage you to visit the Math Help Room
in Carver~385 for additional help.
Open 9:00--16:00 Monday through Friday the Math Help Room
has tutors specifically intended to address Math~265 questions.

\section{Disabilities, academic misconduct, disruptive behavior}
The Department of Mathematics complies with the 
\href{http://www.ada.gov}{American Disabilities Act} in making reasonable 
accommodations for qualified students with disabilities.  Students with 
special needs should call the 
\href{http://www.dso.iastate.edu/dr}{Office of Student Disability Resources} at
515-294-7220.
You can find the Mathematics Department's policies on 
academic misconduct, disruptive behavior, and makeup exams at
\burl{http://www.math.iastate.edu/Faculty/ClassPolicies.html}.

\end{document}
