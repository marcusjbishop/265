\documentclass[12pt]{article}
\usepackage{multicol,graphicx}
\usepackage[colorlinks,breaklinks,linkcolor=red,citecolor=blue]
{hyperref} 
\def\sectionautorefname~#1\null{\S#1\null}
\usepackage{charter,amsmath,amssymb,breakurl}
\usepackage{eulervm}
\usepackage[letterpaper,margin=.75in]{geometry}
\def\equationautorefname~#1\null{(#1)\null}
\def\itemautorefname~#1\null{(#1)\null}
\title{Some Worksheet 3 Solutions}
\author{}\date{}
\let\ln\relax\DeclareMathOperator{\ln}{\mathsf{ln}}
\let\sin\relax\DeclareMathOperator{\sin}{\mathsf{sin}}
\let\arctan\relax\DeclareMathOperator{\arctan}{\mathsf{arctan}}
\let\cos\relax\DeclareMathOperator{\cos}{\mathsf{cos}}
\let\sec\relax\DeclareMathOperator{\sec}{\mathsf{sec}}
\let\min\relax\DeclareMathOperator*{\min}{\mathsf{min}}
\let\max\relax\DeclareMathOperator*{\max}{\mathsf{max}}
\let\sup\relax\DeclareMathOperator*{\sup}{\mathsf{sup}}
\let\inf\relax\DeclareMathOperator*{\inf}{\mathsf{inf}}
\let\lim\relax\DeclareMathOperator*{\lim}{\mathsf{lim}}
\everymath{\displaystyle}
\begin{document}
\maketitle
\thispagestyle{empty}

\begin{enumerate}
\item
\begin{enumerate}
\item $\int_{-1}^1\int_0^1\left(y+1\right)e^{x\left(y+1\right)}dxdy
=\int_{-1}^1\left.e^{x\left(y+1\right)}\right|_0^1\;dy
=\int_{-1}^1\left(e^{y+1}-1\right)dy=e^2-3$
\end{enumerate}
\item
\begin{enumerate}
\item $\int_0^1\int_0^xe^{x^2}dydx
=\int_0^1xe^{x^2}dx=\frac{e-1}{2}$
\item $\int_0^{\sqrt[3]{\pi}}\int_0^xx^4\cos\left(x^2y\right)\;dydx
=\int_0^{\sqrt[3]{\pi}}x^2\cos\left(x^3\right)dx=\frac{2}{3}$
\item $\int_0^4\int_0^{\sqrt{4-y}}\frac{xe^{2y}}{4-y}\;dxdy
=\int_0^4\frac{e^{2y}}{2}dy=\frac{e^8-1}{4}$
\end{enumerate}
\item
\begin{enumerate}
\item
\item $\int_{-1}^1\int_{-\sqrt{1-x^2}}^{\sqrt{1-x^2}}
\left(12+x+y\right)dydx
=\cdots
=\int_{-1}^1\left(24\sqrt{1-x^2}+2x\sqrt{1-x^2}\right)dx$

You can evaluate the first integral using the trig substitution $x=\sin{t}$
and the second integral using the fact
that the integrand is odd, so the integral is zero.
You can also make the substitution $u=1-x^2$ to evaluate the second integral.
The first integral (and hence the answer to the problem) is $12\pi$.
\end{enumerate}
\end{enumerate}
\end{document}
