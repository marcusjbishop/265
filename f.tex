\documentclass[12pt]{exam}
\usepackage[margin=1in]{geometry}
\usepackage{amsmath,amssymb}

\newcommand{\class}{Math 265}
\newcommand{\term}{Spring 2015}
\newcommand{\examnum}{Final Exam}
\newcommand{\examdate}{Makeup Version}
\newcommand{\timelimit}{120 Minutes}
\newcommand{\vect}[1]{\overrightarrow{#1}}
\newcommand{\ve}[1]{{\bf #1}}
\newcommand{\pd}[2]{\frac{\partial #1}{\partial #2}}
\newcommand{\der}[2]{\frac{d #1}{d #2}}
\newcommand{\vcomp}[3]{#1 \ve{i}+ #2 \ve{j} + #3\ve{k}}
\newcommand{\vcomps}[2]{#1 \ve{i}+ #2 \ve{j}}

\pagestyle{head}
\firstpageheader{}{}{}
\runningheader{\class}{\examnum\ - Page \thepage\ of \numpages}{\examdate}
\runningheadrule

\begin{document}

\noindent
\begin{tabular*}{\textwidth}{l @{\extracolsep{\fill}} r @{\extracolsep{6pt}} l}
\textbf{\class} & \textbf{Name:} & \makebox[2in]{\hrulefill}\\
\textbf{\term} &&\\
\textbf{\examnum} &&\\
\textbf{\examdate} &&\\
\textbf{Time Limit: \timelimit} & Section & \makebox[2in]{\hrulefill}
\end{tabular*}\\
\rule[2ex]{\textwidth}{2pt}

{\bf This test is closed book and closed notes. A 
(graphing) calculator is allowed for this test but 
cannot also be a communication device (i.e., your 
iPhone is not a calculator). Answer each question 
completely using \underbar{exact values} unless 
otherwise indicated. Show your work (legibly); answers 
without work and/or justifications will not receive 
credit. Place your final answer in the space provided. 
This exam contains \numpages\ pages (including this 
cover page) and \numquestions\ questions. Show all of 
your work for full credit. }

\begin{center}
Grade Table (for teacher use only)\\
\addpoints
\gradetable[v][questions]
\end{center}

\noindent
\rule[2ex]{\textwidth}{2pt}

\newpage

\begin{questions}

\question[15] 
\begin{parts}
\part Find an equation for the plane containing the
points $P\left(1,0,3\right)$, $Q\left(0,4,2\right)$,
and $R\left(1,1,1\right)$.
\vfill
\uplevel{\makebox[\textwidth]{Equation:\enspace\hrulefill}}
\part Find the area of the triangle $\triangle PQR$.
\vfill
\uplevel{\makebox[\textwidth]{Area:\enspace\hrulefill}}
\end{parts}
\newpage

\question[15]
The trajectory of a moving object is
given by $\mathbf{r}\left(t\right)
=\left\langle e^t\cos{t},e^t\sin{t},e^t\right\rangle$.
Calculate the tangential and normal components 
$a_T$ and $a_N$ of its
acceleration $\mathbf{r}''\left(t\right)$. You need
not calculate $\mathbf{r}''\left(t\right)$,
$\mathbf{T}\left(t\right)$, or $\mathbf{N}\left(t\right)$.
\vfill
\uplevel{\makebox[\textwidth]
{$a_T$:\enspace\hrulefill\;$a_N$:\enspace\hrulefill}}
\newpage

\question[15]
Consider the plane $\Pi$ defined by
$\sin{xyz}=\frac{1}{2}$ and the 
point $P=\left(\pi,1,\frac{1}{6}\right)$.
\begin{parts}
\part Calculate an equation for the tangent plane to $\Pi$
at $P$.
\vfill
\uplevel{\makebox[\textwidth]{Equation:\enspace\hrulefill}}
\part Calculate an equation for the normal line to $\Pi$
through $P$.
\vfill
\uplevel{\makebox[\textwidth]{Equation:\enspace\hrulefill}}
\end{parts}
\newpage

\question[15] Find and classify all critical points
of  $f\left(x,y\right)=x^4-4xy+2y^2$ as local maxima,
local minima, or saddle points.
\vfill
\begin{tabular}{|r|p{1.5in}|p{1.5in}|p{1.5in}|}
\hline\rule{0pt}{24pt}Point&&&\\\hline
Classification\rule{0pt}{24pt}&&&\\\hline
\end{tabular}
\newpage

\question[15] Evaluate 
$\displaystyle\int_0^4\int_{\sqrt{x}}^2\frac{x}{y^5+1}dydx$
by changing the order of integration.
\vfill
$\displaystyle\int_0^4\int_{\sqrt{x}}^2\frac{x}{y^5+1}dydx
=\underline{\int\hspace{1cm}\int\hspace{1in}dxdy}
=\underline{\phantom{\int}\hspace{1in}}$
\newpage

\question[15] Find the mass and center of mass of a
thin plate of density $\delta(x,y)=x+y$
bounded by the lines $y=0$, $y=1-x$, and $y=1+x$.
\vfill
\uplevel{\makebox[\textwidth]
{Mass:\enspace\hrulefill\;
$\overline{x}$:\enspace\hrulefill\;
$\overline{y}$:\enspace\hrulefill}}
\newpage

\question[15] Let $\ve{F}=(xy+y^2)\ve{i}+(x^2-y^2)\ve{j}$ and 
let $C$ be the boundary of the portion of the unit disk
$x^2+y^2\le 1$ lying in the first quadrant.
Use Green's Theorem to find the following.
\begin{parts}
\part The counterclockwise circulation
$\displaystyle\oint_C\ve{F}\cdot\ve{T}\,ds$
\vfill
\uplevel{\makebox[\textwidth]
{$\displaystyle\oint_C\ve{F}\cdot\ve{T}\,ds=$\enspace\hrulefill}}

\part The outward flux $\displaystyle\oint_C\ve{F}\cdot\ve{n}\,ds$
\vfill
\uplevel{\makebox[\textwidth]
{$\displaystyle\oint_C\ve{F}\cdot\ve{n}\,ds=$\enspace\hrulefill}}

\end{parts}

\newpage

\question[15] Let $\ve{F}=2z\ve{i}-4x\ve{j}+3y\ve{k}$ and let $S$ 
be the cap of the sphere $x^2+y^2+z^2=169$ above the plane $z=12$.
Evaluate $\displaystyle\iint_S\nabla\times \ve{F}\cdot\ve{n}\,d\sigma$.
\vfill
\uplevel{\makebox[\textwidth]
{$\displaystyle\iint_S\nabla\times
\ve{F}\cdot\ve{n}\,d\sigma=$\enspace\hrulefill}}

\end{questions}
\end{document} 
