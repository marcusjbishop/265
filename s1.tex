\documentclass[11pt]{article}
\usepackage{booktabs}
\usepackage[euler-digits]{eulervm}
\usepackage[colorlinks=true, linkcolor=blue, breaklinks=true]{hyperref} 
\def\sectionautorefname~#1\null{\S#1\null}
\usepackage{charter,amsmath,amssymb,breakurl}
\usepackage[letterpaper,margin=.9in]{geometry}
\title{Syllabus for Math~166, Section 29}
\author{Dr Marcus Bishop, Iowa State University}
\begin{document}
\maketitle

\section{Time, place}\label{Time}
We will meet at 15:10--16:00
on Mondays, Tuesdays, Thursdays, and Fridays
in Carver~18.
The midterm exam is at 20:15--21:45
on Wednesday 8 October in Coover~2245.
You should plan your schedule accordingly.

\section{Credits, requirements, prerequisites, materials}\label{Require}
The course is worth $4$~credits.
To take this course, you should have taken Math~165
or the equivalent. You will need the twelfth edition of 
\href{http://wps.aw.com/aw_thomas_calculus_series}
{\em Thomas' Calculus, Early Trancendentals}
by Thomas, Weir, and Hass, hereafter referenced by
\href{http://wps.aw.com/aw_thomas_calculus_series}{Thomas}.
You will also want a 
\href{http://en.wikipedia.org/wiki/TI-83}{TI-83} or
\href{http://en.wikipedia.org/wiki/TI-84}{TI-84}
graphing calculator. Unfortunately, we cannot allow you to use
any device with algebraic capabilities greater than those of the TI-83
or TI-84 on quizzes and exams. However, we encourage
you to supplement your learning with a computer algebra system
of your choice.
In class I will use \href{http://www.sagemath.org}{\textsf Sage},
an open-source computer algebra system that you can install on your
computer for free if you wish.

\section{Instructor information} You may visit me during my office 
hours at 14:00--15:00 on Mondays, Tuesdays, Thursdays, and Fridays
in Carver~418. If my office hours are inconvenient for you
please contact me at 
\href{mailto:mbishop@iastate.edu}{\tt mbishop@iastate.edu} or at 
515-294-0027 to arrange an alternate time.
You can also visit the Math Help Room (see \autoref{MathCenter}).

\section{Course goals, content}
Calculus is the starting point for
many different University programs.
This is because calculus has proved to be extremely useful
in a variety of different areas due to
the fact that these areas all
deal in one way or another with {\em change},
the central theme of calculus.
Newton and Leibniz simultaneously discovered calculus in the 
seventeenth century in the context of theoretical physics.
Newton was interested in the motion of planets,
that is, the {\em change} of the positions of planets.
However, over the subsequent centuries, it was discovered
that calculus could be applied more generally to
{\em all} quantities that change, having applications in life sciences,
actuarial science, economics, computer science, and medicine,
to name a few.

Practically speaking, calculus can be applied in situations
where one wants to study the change in some quantity
that can be expressed as (or at least approximated by)
a mathematical function, like a polynomial or an exponential function.
For this reason, much of Math~165 was concerned with
{\em modeling} changing quantities using mathematical functions,
a task which is also useful outside the context of calculus.
In Math~165 we also introduced the central
operations of calculus, namely, the limit, the derivative, and the integral,
with special emphasis placed on the techniques used to calculate
the derivative.

Now in Math~166 we continue the program, beginning with the
integral, with special emphasis placed on the techniques used to
integrate functions. In contrast with differentiation techniques,
we will see that integration techniques range from
straightforward to subtle. Nevertheless, the ability to integrate
a large class of functions translates to the ability to solve a large
class of real world problems. We will also introduce series, which
provide alternate representations of differentiable functions.
Finally, we will introduce parametric equations and polar
coordinates. These provide alternate ways to look at problems
and often lead to more convenient computations.

\section{Course Objectives}
\subsection{Applications of the Integral} 
Set up and evaluate integrals to calculate
  \begin{enumerate}
    \item Area of a plane region
    \item Volume of a solid of revolution
    \item Length of a plane curve
    \item Area of a surface of revolution
    \item Work done by a variable force
    \item Moments and center of mass of a plane lamina, centroid of a plane
          region. 
  \end{enumerate}
\subsection{Techniques of Integration}
  \begin{enumerate}
    \item Evaluate integrals of trigonometric functions.
    \item Use trigonometric and rationalizing substitutions to evaluate
          integrals. 
    \item Carry out integration by parts and apply it to evaluate integrals.
    \item Use partial fractions to evaluate integrals of rational functions.
  \end{enumerate}
\subsection{Indeterminate Forms and Improper Integrals}
  \begin{enumerate}
    \item Apply l'Hospital's Rule to evaluate limits having the
          indeterminate forms $0/0$, $\infty/\infty$, $0\cdot\infty$ and
          $\infty-\infty$. 
    \item Determine convergence or divergence of improper integrals; evaluate
          improper integrals that converge.
  \end{enumerate}
\subsection{Infinite Series}
  \begin{enumerate}
    \item Apply limit rules to calculate limits of sequences. Apply the concept
          of boundedness to identify convergent monotonic sequences.
    \item Use the concept of partial sum to distinguish between convergent and
          divergent series and to define the sum of a convergent series.
    \item Recognize geometric series and collapsing series and calculate their
          sums when convergent.
    \item Use the integral test, the comparison test, the limit comparison
          test and the ratio test to determine the convergence or divergence of
          series. Use the error estimate derived from the integral test to
          estimate sums or tails of series.
    \item Recognize alternating series, and apply the alternating series test
          and associated error estimate. Identify absolutely convergent series.
    \item Determine radius of convergence and convergence set of a power
          series. 
    \item Apply term-by-term integration and differentiation to power series.
          Perform algebraic operations on power series.
    \item Expand a function in a Taylor series. Recall and use the Taylor
          series of elementary functions.
    \item Use the remainder in Taylor's formula to estimate the approximation
          error in a Taylor polynomial.
  \end{enumerate}
\subsection{Plane Parametric Curves, Polar Coordinates}
  \begin{enumerate}
    \item Derive parametric representations for plane curves described
          ``mechanically'' (for example, the cardiod). 
    \item Find tangents and compute length for parametric curves.
    \item Use polar coordinates, and convert between polar and rectangular
          coordinates. Identify the polar equations for lines, circles and
          conics. 
    \item Compute the area of regions whose boundaries are defined by
          equations in polar coordinates.
  \end{enumerate}

\section{Calculation of grades}\label{Assessment}
Your grade will be determined by the results of 
26 online assignments (see \autoref{Online}),
four quizzes, two hour exams, one midterm exam, and one 
final exam.
%In order to accommodate unforeseen events that would cause you
%to pass up the opportunity to take quizzes or submit assignments on time,
%we will exclude from your final grade
%the quiz with the lowest score among all your quizzes
%and the two written assignments with the lowest scores among
%all your written assignments.
Your final grade will be determined from the total
number of points you receive in the course using the schedule below,
where the final exam accounts for 250~points,
the hour and midterm exams for 150~points each,
the quizzes for 100~points,
and the online assignments for 200~points.
The following table shows the {\em minimum} final grade you will receive
if your point total falls in the corresponding range.
\begin{center}\begin{tabular}{clclcl}
&&A&930--1000
&A--&900--929\\
B+&870--899\qquad\qquad
&B&830--869\qquad\qquad
&B--&800--829\\
C+&770--799
&C&730--769
&C--&700--729\\
D+&670--699
&D&600--669\\
&&F&0--599
\end{tabular}\end{center}
All the scores mentioned above will be available to you in
\href{https://bb.its.iastate.edu}{Blackboard}. Using your scores
and the formula above, you should be able to calculate your own final grade,
and even estimate your final grade midway through the semester.
It is therefore unnecessary for you to ever ask the instructor
what your grade in the course is, which we kindly ask you to refrain from
doing.

We emphasize that the grades will be calculated in the manner described
above, not ``curved''. However, in the unlikely event that an exam or quiz question is determined
to be unfair or overly difficult, it will be dropped from the total score of
every student.

\section{Online exercises}\label{Online}
The online exercises will be delivered through the
\href{http://iastate.mylabsplus.com}{MyLabsPlus} system.
Your subscription to 
\href{http://iastate.mylabsplus.com}{MyLabsPlus}
comes with the purchase of your
\href{http://wps.aw.com/aw_thomas_calculus_series}{textbook}.
However, if you purchased the book from some other source,
then you can purchase a 
\href{http://iastate.mylabsplus.com}{MyLabsPlus}
subscription at the bookstore or directly at
\href{http://iastate.mylabsplus.com}{MyLabsPlus}.
To access
\href{http://iastate.mylabsplus.com}{MyLabsPlus}
sign in at
\begin{center}
\href{http://iastate.mylabsplus.com}{\tt http://iastate.mylabsplus.com}
\end{center}
with your NetID and password.
You should also run the
\href{https://www.mathxl.com/BrowserCheck/BrowserCheck.aspx?appproductid=3&courseid=2744761&handler_urn=pearson%2fmlp_mml_xl%2fslink%2fx-pearson-mlp_mml_xl&productid=ccng}{Browser Checker}
to ensure that all the exercises will appear correctly.
In addition to providing your online assignments
\href{http://iastate.mylabsplus.com}{MyLabsPlus}
also contains a number of useful resources, not the least of which is an
electronic copy of the entire text, which could obviate
the somewhat strenuous task of transporting the book.

You will have 26~online assignment corresponding
approximately with the sections of 
\href{http://wps.aw.com/aw_thomas_calculus_series}{Thomas}
that we will cover. Each online assignment will be available while we cover the
corresponding section in class and must be completed
before 23:59 of the date shown on the assignment, which
is usually one class period after we cover the section in class.
In this way you can raise questions about
the assignment in class before completing the assignment.

\section{Expectations, suggestions} Naturally we expect you to attend 
class meetings, complete online assignments on 
time, prepare for quizzes and exams, and participate in classroom 
activities. We also strongly encourage you to read
\href{http://wps.aw.com/aw_thomas_calculus_series}{Thomas}
in addition to attending lectures.
In addition to being generally prudent to 
supplement your learning materials, reading the text has a number of 
advantages over attending lectures alone. Namely, the text is beautifully 
typeset and edited by professionals, very clearly written with readers 
of exactly your level in mind, and virtually free of mistakes.

\section{Course schedule}\label{Schedule} While the exact subject
matter to be covered in class shown in the following schedule is
subject to change slightly, the quizzes and exams will be conducted
on {\em exactly} the dates shown.

\begin{tabular}{c|cl|cl|cl|cl}
&\multicolumn{2}{c|}{\bf Monday}
&\multicolumn{2}{c|}{\bf Tuesday}
&\multicolumn{2}{c|}{\bf Thursday}
&\multicolumn{2}{c}{\bf Friday}\\
{\bf Week}&{\bf Date}&{\bf In class}
&{\bf Date}&{\bf In class}&{\bf Date}&{\bf In class}
&{\bf Date}&{\bf In class}\\\toprule
1&25 Aug&\S6.1&26 Aug&\S6.1&28 Aug&\S6.1&29 Aug&\S6.2\\\midrule
2&1 Sep&Holiday&2 Sep&\S6.2&4 Sep&\S6.3&5 Sep&\S6.3, {\bf Quiz 1}\\\midrule
3&8 Sep&\S6.4&9 Sep&\S6.4&11 Sep&\S6.5&12 Sep&\S6.5\\\midrule
4&15 Sep&\S6.6&16 Sep&\S6.6&18 Sep&Review&19 Sep&{\bf Exam 1}\\\midrule
5&22 Sep&\S8.1&23 Sep&\S8.1&25 Sep&\S8.2&26 Sep&\S8.2\\\midrule
6&29 Sep&\S8.3&30 Sep&\S8.3&2 Oct&\S8.4&3 Oct&\S8.4, {\bf Quiz 2}\\\midrule
7&6 Oct&Review&7 Oct&&9 Oct&\S8.5&10 Oct&\S8.5\\\midrule
8&13 Oct&\S8.7&14 Oct&\S8.7&16 Oct&\S10.1&17 Oct&\S10.1\\\midrule
9&20 Oct&\S10.2&21 Oct&\S10.3&23 Oct&\S10.3&24 Oct&\S10.4, {\bf Quiz 3}\\\midrule
10&27 Oct&\S10.5&28 Oct&\S10.5&30 Oct&\S10.6&31 Oct&\S10.7\\\midrule
11&3 Nov&\S10.7&4 Nov&\S10.8&6 Nov&Review&7 Nov&{\bf Exam 3}\\\midrule
12&10 Nov&\S10.8&11 Nov&\S10.9&13 Nov&\S10.9&14 Nov&\S10.10\\\midrule
13&17 Nov&\S10.10&18 Nov&\S11.1&20 Nov&11.1&21 Nov&\S11.2, {\bf Quiz 4}\\\midrule
14&1 Dec&\S11.2&2 Dec&\S11.3&4 Dec&\S11.4&5 Dec&\S11.5\\\midrule
15&8 Dec&\S11.6&9 Dec&\S11.6&11 Dec&\S11.7&12 Dec&Review\\\midrule
\end{tabular}

\section{Math Help Room}\label{MathCenter}
We strongly encourage you to visit the Math Help Room
in Carver~385 for additional help.
Open 9:00--16:00 Monday through Friday the Math Help Room
has tutors specifically intended to address Math~166 questions.

\section{Students with disabilities, academic honesty, disruptive behavior}
The Department of Mathematics complies with the 
\href{http://www.ada.gov}{American Disabilities Act} in making reasonable 
accommodations for qualified students with disabilities.  Students with 
special needs should call the 
\href{http://www.dso.iastate.edu/dr}{Office of Student Disability Resources} at
515-294-7220.
We strictly enforce the University's policies on 
\href{http://www.dso.iastate.edu/ja/academic/misconduct}{academic misconduct}
and \href{http://www.dso.iastate.edu/sa/issuesconcerns/disruption}
{disruptive behavior}.

\end{document}
