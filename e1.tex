\documentclass[answers,12pt,addpoints]{exam}
\usepackage{graphicx,multicol}
\usepackage{charter,amsmath,amssymb}
\usepackage{eulervm}
\usepackage[letterpaper,margin=1in]{geometry}
\pagestyle{headandfoot}
\runningheadrule
\runningheader{\bf Math 265}
{\bf Exam One, Page \thepage\ of \numpages}
{\bf 6 January 2015}
\firstpagefooter{}{}{}
\runningfooter{}{}{}
\let\cos\relax\DeclareMathOperator{\cos}{\mathsf{cos}}
\let\sin\relax\DeclareMathOperator{\sin}{\mathsf{sin}}
\let\ln\relax\DeclareMathOperator{\ln}{\mathsf{ln}}
\let\lim\relax\DeclareMathOperator*{\lim}{\mathsf{lim}}
\everymath{\displaystyle}
\begin{document}
\begin{center}
\huge Math 265 Exam One\\
\LARGE Spring 2015
\end{center}

\ifprintanswers\else
\begin{center}
\fbox{\fbox{\parbox{5.5in}{
This exam has \numquestions~questions.
It has been printed on \numpages~pages and is worth \numpoints~points.
Answer all the questions in the spaces provided.
%The last page has been left blank for your calculations.
Although you may use any calculator on this exam,
you must clearly indicate how you arrived at your answers
in order to receive maximum credit.
By signing below, you pledge that you
\begin{enumerate}
\item will not communicate to any person in any conceivable way anything
about the contents of this exam
until all students have taken it, and
\item have not been the recipient of such communication from anyone else.
\end{enumerate}}}}
\end{center}
\vspace{.2in}
\makebox[\textwidth]{Your signature:\enspace\hrulefill}\\
\vspace{.2in}\\
\makebox[\textwidth]{Your name:\enspace\hrulefill}\\
\begin{center}\gradetable[h][questions]\end{center}
\fi

\begin{questions}

\question[15] What values of $\alpha$ make
$2\alpha\mathbold{i}-\alpha\mathbold{j}+16\mathbold{k}$
orthogonal to $5\mathbold{i}+\alpha\mathbold{j}-\mathbold{k}$?
\begin{solution}[1in]
\begin{align*}
0&=\left\langle 2\alpha,-\alpha,16\right\rangle
\cdot\left\langle 5,\alpha,-1\right\rangle\\
&=10\alpha-\alpha^2-16\\
\Leftrightarrow\qquad
0&=\alpha^2-10\alpha+16\\
\Leftrightarrow\qquad
\alpha&=\frac{10\pm\sqrt{10^2-4\left(16\right)}}{2}\\
&=5\pm 3=2,8
\end{align*}
\end{solution}
\ifprintanswers\else\newpage\fi

\question[40]
Let $P=\left(0,1,0\right)$, $Q=\left(-1,1,2\right)$,
and $R=\left(2,1,-1\right)$ and let $\Pi$ be the plane
containing $P,Q,R$.
\begin{parts}
\part Find the measure of $\angle PQR$.
\begin{solution}[2in]
Observe that $\overrightarrow{QP}=\left\langle
1,0,-2\right\rangle$ and $\overrightarrow{QR}=\left\langle
3,0,-3\right\rangle$ so that
$\left|\overrightarrow{QP}\right|=\sqrt{5}$ and
$\left|\overrightarrow{QR}\right|=\sqrt{18}$.
Then since
$\overrightarrow{QP}\cdot\overrightarrow{QR}
=9$ we have
\[\cos\left(\theta\right)=\frac{9}{\sqrt{5}\sqrt{18}}
=\frac{3}{\sqrt{10}}\]
where $\theta$ is the measure of $\angle PQR$.
Solving for $\theta$ gives approximately $0.32$~radians
or $18^\circ$.
\end{solution}

\part\label{PiPerp} Find a vector perpendicular to $\Pi$.
\begin{solution}[2in]
We can take
$\overrightarrow{QP}\times\overrightarrow{QR}
=\left|\begin{array}{ccc}
\mathbold{i}&\mathbold{j}&\mathbold{k}\\
1&0&-2\\3&0&-3\end{array}\right|=-3\mathbold{j}$.
\end{solution}
\part Find the area of $\triangle PQR$.
\begin{solution}[1.75in]
$\left|\overrightarrow{QP}\times\overrightarrow{QR}\right|
=3$ is the area of the parallelogram formed by
$\overrightarrow{QP}$ and $\overrightarrow{QR}$,
so $\frac{3}{2}$ is the area of $\triangle PQR$.
\end{solution}
\part Find the projection of $\overrightarrow{PQ}$
onto $\overrightarrow{PR}$.
\begin{solution}[1.75in]
$\overrightarrow{PQ}=\left\langle -1,0,2\right\rangle$
and $\overrightarrow{PR}=\left\langle 2,0,-1\right\rangle$, so
\[\mathsf{proj}_{\overrightarrow{PR}}\overrightarrow{PQ}
=\frac{\overrightarrow{PQ}\cdot\overrightarrow{PR}}
{\left|\overrightarrow{PR}\right|^2}\overrightarrow{PR}
=\frac{-4}{5}\left\langle 2,0,-1\right\rangle.\]
\end{solution}

\item\label{PiEquation} Find an equation for $\Pi$.
\begin{solution}[1.75in]
Knowing that $\mathbold{j}$ is normal to
$\Pi$ by (\ref{PiPerp})
and $P=\left(0,1,0\right)$ is on $\Pi$, we can take
$y=1$ as an equation for $\Pi$.
\end{solution}

\item Find the distance from $\Pi$ to the origin.
\begin{solution}[1.75in]
It should be clear from (\ref{PiEquation}) that $1$
is the distance from $\Pi$ to the origin.
\end{solution}

\end{parts}

\question[20] Find an equation for the line
tangent to the logarithmic spiral
\[\mathbold{r}\left(t\right)
=\left\langle e^t\cos{t},e^t\sin{t},e^t\right\rangle\]
at the point $\left(1,0,1\right)$.
\begin{solution}[1in]
First note that $t=0$ is the point such that
$\mathbold{r}\left(t\right)=\left(1,0,1\right)$.
Now since
\[\mathbold{r}'\left(t\right)
=\left\langle e^t\cos{t}-e^t\sin{t},
e^t\sin{t}+e^t\cos{t},e^t\right\rangle\]
we have $\mathbold{r}'\left(0\right)
=\left\langle1,1,1\right\rangle$ so that
the tangent line at $t=0$ is given by
\[\left\langle 1,0,1\right\rangle+t\left\langle
1,1,1\right\rangle.\]
\end{solution}
\ifprintanswers\else\newpage\fi

\question[25] Find the intersection of the planes
$x-y-z=-1$ and $x-y+z=1$.
\begin{solution}[1in]
It's helpful to observe that since
$\mathbold{n}_1=\left\langle 1,-1,-1\right\rangle$
is normal to the first plane and
$\mathbold{n}_2=\left\langle 1,-1,1\right\rangle$
is normal to the second plane,
the vector $\mathbold{n}_1\times\mathbold{n_2}
=\left\langle -2,-2,0\right)$ is parallel to the
intersection of the two planes.
Thus the line of intersection is parallel
to $\left\langle 1,1,0\right\rangle$.
By inspection we can see that $\left(0,0,1\right)$
is on the line of intersection, being on both planes.
Thus the intersection is given by
$\left\langle 0,0,1\right\rangle
+t\left\langle 1,1,0\right\rangle$.

\end{solution}

\end{questions}
\end{document}
