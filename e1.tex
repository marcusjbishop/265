\documentclass[answers,addpoints]{exam}
\usepackage{graphicx,multicol}
\usepackage{charter,amsmath,amssymb}
\usepackage{eulervm}
\usepackage[letterpaper,margin=1in]{geometry}
\pagestyle{headandfoot}
\runningheadrule
\runningheader{\bf Math 265}
{\bf Exam One, Page \thepage\ of \numpages}
{\bf 6 January 2015}
\firstpagefooter{}{}{}
\runningfooter{}{}{}
\let\cos\relax\DeclareMathOperator{\cos}{\mathsf{cos}}
\let\sin\relax\DeclareMathOperator{\sin}{\mathsf{sin}}
\let\ln\relax\DeclareMathOperator{\ln}{\mathsf{ln}}
\let\lim\relax\DeclareMathOperator*{\lim}{\mathsf{lim}}
\everymath{\displaystyle}
\begin{document}
\begin{center}
\huge Math 265 Exam One\\
\LARGE Spring 2015
\end{center}

\ifprintanswers\else
\begin{center}
\fbox{\fbox{\parbox{5.5in}{
This exam has \numquestions~questions.
It has been printed on \numpages~pages and is worth \numpoints~points.
Answer all the questions in the spaces provided.
The last page has been left blank for your calculations.
Although you may use any calculator on this exam,
you must clearly indicate how you arrived at your answers
in order to receive maximum credit.
By signing below, you pledge that you
\begin{enumerate}
\item will not communicate to any person in any conceivable way anything
about the contents of this exam
until all students have taken it, and
\item have not been the recipient of such communication from anyone else.
\end{enumerate}}}}
\end{center}
\vspace{.2in}
\makebox[\textwidth]{Your signature:\enspace\hrulefill}\\
\vspace{.2in}\\
\makebox[\textwidth]{Your name:\enspace\hrulefill}\\
\vfill
\begin{center}\gradetable[h][questions]\end{center}
\vfill
\fi

\begin{questions}

\question[12] What values of $a$ make
$\mathbold{v}=2a\mathbold{i}-a\mathbold{j}+16\mathbold{k}$
orthogonal to $5\mathbold{i}+a\mathbold{j}-\mathbold{k}$?

\question[20]
Let $P=\left(0,1,0\right)$, $Q=\left(-1,1,2\right)$,
and $R=\left(2,1,-1\right)$ and let $\Pi$ be the plane
containing $P,Q,R$.
\begin{parts}
\part Find the measure of $\angle PQR$.
\part Find the area of $\Delta PQR$.
\part Find a vector perpendicular to $\Pi$.
\item Find an equation for $\Pi$.
\item Find the distance from $\Pi$ to the origin.
\end{parts}

\question[20] One force is pushing an object in a direction
$50^\circ$ south of east with a force of 25~newtons.
A second force is simultaneously pushing the object
in a direction $70^\circ$ north of west with a force of
60~newtons. If the object is to remain stationary,
give the direction and magnitude of the third force that must
be applied to the object to counterbalance
the first two.


\end{questions}
\end{document}
